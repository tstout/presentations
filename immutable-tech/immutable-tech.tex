% 
% Compile with ./build.sh
% 

\documentclass{beamer}

\usetheme{Madrid}    

\usepackage{listings}                                   
\usepackage{hyperref}
\usepackage{graphicx}                                 
\usepackage{tabularx}
\usepackage{microtype}
\usepackage[T1]{fontenc}
\usepackage[scaled]{beramono}
\usepackage{minted}
\usepackage{xcolor}
\usepackage{pgfplots}
\usepackage{dirtytalk}
\usepackage{tikz}
\usetikzlibrary{tikzmark,fit}

%\usepackage{enumitem}
\pgfplotsset{compat=1.6} 

\newcommand\Small{\fontsize{5}{5.2}\selectfont}
\newcommand*\LSTfont{\Small\ttfamily\SetTracking{encoding=*}{-60}\lsstyle}
\renewcommand{\footnotesize}{\tiny}

\hypersetup{colorlinks=color, linkcolor=black}
\definecolor{OliveGreen}{rgb}{0,0.6,0}
\graphicspath{{./images/}}
% 
% Turn off beamer nav stuff...
% 
\setbeamertemplate{navigation symbols}{}


%\input{lst-config/clojure-config}
\begin{document}

\begin{frame}
  \frametitle{Invest In Learning Immutable Tech}
  \center{
    %
    % Graphic for Title Page
    %
    \includegraphics[scale=.15]{tech2}
    
  }
\end{frame}

\frame{
  \frametitle{Change is Inevitable}
  \say{To improve is to change; to be perfect is to change often.}

  \rightline{{\rm  -- Winston Churchill}}
}

\frame{
  \frametitle{Change is Inevitable?}
  \say{The more things change, the more they stay the same.}

  \rightline{{\rm  -- Jean-Baptiste Alphonse Karr}}
}


\frame{
  \frametitle{Title Page 2}
  % \begin{columns}
  %   \begin{column}{.49\textwidth}
  %     \includegraphics[scale=.50]{church}
  %   \end{column}
  %   \begin{column}{.49\textwidth}
  %     \itemize{
  %     \item Alonzo Church - 1936
  %     \item $\lambda$ Calculus
  %      \footnote{\href{https://www.ics.uci.edu/~lopes/teaching/inf212W12/readings/church.pdf}
  %        {An Unsolvable Problem of Elementary Number Theory}}
  %     }

  %   \end{column}
  %\end{columns}
}

%\resetcounter[footnote]
%\setcounter{footnote}{0} 


%
% For databases, mention stack-overflow and wikipedia as examples where 
% relational can be web scale
%
\frame{
  \frametitle{Database}
  \begin{columns}
    \begin{column}{.49\textwidth}
      \includegraphics[scale=.60]{db}
    \end{column}
    \begin{column}{.49\textwidth}
      \itemize{
      \item Postgres
      \item SQL Server
      \item Oracle
      \item SQLite
      \item Couchbase
      \item Cassandra
      \item DynamoDB
      \item Bigtable
      \item Snowflake
      \item DB2
      \item H2
      %  \footnote{\href{https://www.ics.uci.edu/~lopes/teaching/inf212W12/readings/church.pdf}
      %    {An Unsolvable Problem of Elementary Number Theory}}
      }

    \end{column}
  \end{columns}
  % \center{
  %   \includegraphics[scale=.75]{db}
    
  }

  \frame{
    \frametitle{Operating Systems}
    \begin{columns}
      \begin{column}{.49\textwidth}
        \includegraphics[scale=.13]{operating-systems}
      \end{column}
      \begin{column}{.49\textwidth}
        \itemize{
        \item Trouble Shooting
        \item Debugging
        \item Performance Tuning
        \item Productivity
        %  \footnote{\href{https://www.ics.uci.edu/~lopes/teaching/inf212W12/readings/church.pdf}
        %    {An Unsolvable Problem of Elementary Number Theory}}
        }
  
      \end{column}
    \end{columns}
    % \center{
    %   \includegraphics[scale=.75]{db}
      
    }


% \frame{
%   \frametitle{Title - Page of Items}
%   \begin{itemize}
%   \item Item 1
%   % To not show following item until keypress
%   % \pause
%   \item Item 2
%   \item Item 3
%   \end{itemize}
% }

% Example of inlining code with minted
% \begin{frame}[fragile]
%   \frametitle{Functions}
%   Typical function definition:
%   \begin{minted}[fontsize=\normalsize,escapeinside=||]{clojure}
%     (defn foo 
%       "optional multiline doc comment"
%       {:optional :metadata-map} 
%       [arg1 arg2]
%       {:optional :pre/post-conditions-map}
%       (...implementation...))
%   \end{minted}

%   \vspace{1 cm}

% %% [autogobble,fontfamily=myfont,escapeinside=||}{c}

%   This is an anonymous function (sum of its args):
%   \begin{minted}[fontsize=\normalsize]{clojure}
%     (fn [a b] (+ a b))   
%   \end{minted}
%   \vspace{1 cm}

%   An abbreviated equivalent:
%   \begin{minted}[fontsize=\normalsize]{clojure}
%     #(+ %1 %2)
%   \end{minted}
% \end{frame}


  \frame{
    \frametitle{Networking}
    \begin{columns}
      \begin{column}{.49\textwidth}
        \includegraphics[scale=.13]{operating-systems}
      \end{column}
      \begin{column}{.49\textwidth}
        \itemize{
        \item TCP
        \item UDP
        \item HTTP
        \item SFTP
        \item SSL
        \item DHCP
        \item DNS
        %  \footnote{\href{https://www.ics.uci.edu/~lopes/teaching/inf212W12/readings/church.pdf}
        %    {An Unsolvable Problem of Elementary Number Theory}}
        }
  
      \end{column}
    \end{columns}
    % \center{
    %   \includegraphics[scale=.75]{db}
      
    }



%https://clojure.org/api/cheatsheet
\frame{
  \frametitle{Resources}
  \begin{itemize}

   \item \href {https://www.example.com/}{\color {blue}{Link to useful info}}

    %% put back ref to source code-[]
  \end{itemize}
}

\end{document}
